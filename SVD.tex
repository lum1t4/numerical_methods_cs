
\documentclass[a4paper, 12pt]{article}
% Packages
\usepackage[utf8]{inputenc} % Input encoding
\usepackage[T1]{fontenc}    % Font encoding
\usepackage[margin=1in]{geometry} % Adjust margins
\usepackage{titling}        % Custom title format
\usepackage{fancyhdr}       % Custom headers and footers
\usepackage{lipsum}         % For generating dummy text (remove in your document)
\usepackage{amsmath,amsfonts,amssymb}
\usepackage[english]{babel}
\usepackage{mathtools}
\usepackage{xcolor}
\usepackage{mdframed}
% Page Style
\pagestyle{fancy}
% Remove indentation at the beginning of a paragraph
\setlength{\parindent}{0pt}
\fancyhf{}
\lhead{\thetitle}
\rhead{\theauthor}
\cfoot{\thepage}
% Title
\title{Notes for Numerical Methods}
\author{Gianluca Calò}
\date{\today}
\begin{document}
\maketitle

\section{2-Norm of a Matrix with SVD}
The 2-norm of the matrix $A$ is equal to $\sigma_1$, the greatest singular value.
\[
\|A\|_2 = \sigma_1
\]
\textbf{Proof}
\[
\begin{aligned}
\|A\|_2 &= \|U\Sigma V^T\|_2 = \|\Sigma\|_2 = \sqrt{\sum_{i=1}^r \sigma_i^2} = \sqrt{\sigma_1^2} = \sigma_1
\end{aligned}
\]
Notes:
\begin{itemize}
    \item The 2-norm of a matrix: $\|A\|_2 = \sqrt{\rho(A^TA)}$
    \item Sigma has a special structure where the singular values are in the main diagonal and are sorted in descending order
    \item $\Sigma^T\Sigma = \begin{pmatrix}
        \sigma_1^2 & \ldots & 0 \\
        \vdots & \ddots & \vdots \\
        0 & \ldots & \sigma_r^2
        \end{pmatrix}$.
\end{itemize}

The same is true for the \textbf{Frobenius Norm}
\[
\|A\|_F = \sqrt{\sum_{i=1}^r \sigma_i^2} = \sigma_1
\]

\section{Low-Rank Approximation problem and Eckart-Young Theorem}
In mathematics, low-rank approximation is a minimization problem, in which the cost function measures
the fit between a given matrix (the data) and an approximating matrix (the optimization variable),
subject to a constraint that the approximating matrix has reduced rank. The problem is used for mathematical
modeling and data compression. The rank constraint is related to a constraint on the complexity of a model that
fits the data. In applications, often there are other constraints on the approximating matrix apart from the rank constraint.

Let $A \in \mathbb{R}^{m \times n}$ with rank $r$ and for $k \leq r$, the matrix approximation problem:
\[
\min_{rank(Z)=k} \|A-Z\|_2
\]
has the solution $$Z_F = A_k = \sum_{i=1}^k \sigma_i u_i v_i^T$$ and the value of the minimum is
\[
    \|A-A_{k}\|_{2}=\left\|\sum _{i=1}^{n}\sigma _{i}u_{i}v_{i}^{\top }-\sum _{i=1}^{k}\sigma _{i}u_{i}v_{i}^{\top }\right\|_{2}=\left\|\sum _{i=k+1}^{n}\sigma _{i}u_{i}v_{i}^{\top }\right\|_{2}=\sigma _{k+1}
\]

Generally we have data that is noisy $\bar{A} = A + N$ where $N$ is the noise matrix and $A$ is the original data matrix.
We observe that the noisy data have full rank $p$ (random matrices are likely to have full rank or a high rank)
and we might know the size of the noise since it may depend from measures or
channel used. We can use the singular value decomposition to find the original data matrix $A$ than has rank $r, r << p$.

\[
\begin{aligned}
\bar{A} &= A + N \\
\bar{A} &= \bar{U}\bar{\Sigma}\bar{V}^T \\
A &\approx \sum_{i=1}^r \sigma_i u_i v_i^T = \bar{A_r} \\
\end{aligned}
\]

where $\sigma_{r+1}, \ldots, \sigma_p$ are very small, representing the size of the noise.
\section*{Principal Component Regression}
Solution of least square problem when the matrix $A$ has rank $r << p$.

To solve this problem, we use the singular value decomposition and since the solution is not unique,
we choose the one of minimal norm.
\[
A = U \Sigma V^T =
\begin{pmatrix} U_1 & U_2 \end{pmatrix}
\begin{pmatrix}
\Sigma_1 & 0 \\
0 & 0 
\end{pmatrix} \begin{pmatrix}
V_1^T \\
V_2^T
\end{pmatrix}
\]

$\Sigma_1 = \text{diag}(\sigma_1, \ldots, \sigma_r)$ and $\Sigma$ is a square non-singular $r \times r$ matrix.

We use this factorization to solve the least squares problem
\[
    \min_x \| Ax - b \|_2^2
\]
This can be written as
\[
\begin{aligned}
    \min_x \| Ax - b \|_2^2 &=
    \left\|
        \begin{pmatrix} U_1 & U_2 \end{pmatrix}
        \begin{pmatrix} \Sigma_1 & 0 \\ 0 & 0 \end{pmatrix}
        \begin{pmatrix} V_1^T \\ V_2^T \end{pmatrix}
        x - b
    \right\|_2^2 \\
    & = \left\|
        U \begin{pmatrix}
        \Sigma_1 & 0 \\
        0 & 0 
        \end{pmatrix}
        \begin{pmatrix}
            V_1^T \\
            V_2^T
        \end{pmatrix}
        x - UU^T b
    \right\|_2^2 \\
    & = \left\|
    \begin{pmatrix}
        \Sigma_1 & 0 \\
        0 & 0 
        \end{pmatrix}
        \begin{pmatrix}
            V_1^T \\
            V_2^T
        \end{pmatrix}
        x - 
        \begin{pmatrix}
            U_1^T \\
            U_2^T
        \end{pmatrix}b
    \right\|_2^2 \\
    & = \left\|
        \begin{pmatrix}
            \Sigma_1 V_1^T x - U_1^T b\\
            - U_2^T b
        \end{pmatrix}
    \right\|_2^2 \\
\end{aligned}
\]

Let $y = V^T x$:
\[
y = \begin{pmatrix}
V_1^Tx \\
V_2^Tx
\end{pmatrix} = \begin{pmatrix}
y_1 \\
y_2
\end{pmatrix}
\]
And we have since we can split the norm in two parts being the sum of squares
\[
\min_x \| Ax - b \|_2^2 = \| \Sigma_1 y_1 - U_1^T b \|_2^2 + \| U_2^T b \|_2^2
\]

The minimum is reached when 
$$\| \Sigma_1 y_1 - U_1^T b \|_2^2 = 0 \implies \Sigma_1 y_1 = U_1^T b \implies y_1 = \Sigma_1^{-1} U_1^T b$$
and it is
\[
\min_x \| Ax - b \|_2 = \| U_2^T b \|_2
\]

We need to choose $y_2$ to find $x$. We would to minimize the $\| x \|_2$ equal to $\| y \|_2 = \| y_1 \|_2 + \| y_2 \|_2$ so
we choose $y_2 = 0$.
$$
\begin{aligned}
    x &= Vy = V\begin{pmatrix} \Sigma_1^{-1} U_1^T b \\ 0 \end{pmatrix} = \begin{pmatrix} V_1 & V_2 \end{pmatrix} \begin{pmatrix} \Sigma_1^{-1} U_1^T b \\ 0 \end{pmatrix} = V_1 \Sigma_1^{-1} U_1^T b \\
    x &= V \begin{pmatrix} \Sigma_1^{-1} & 0 \\ 0 & 0 \end{pmatrix} U^T b
\end{aligned}
$$

The matrix $V \begin{pmatrix} \Sigma_1^{-1} & 0 \\ 0 & 0 \end{pmatrix} U^T$ is the \textbf{Moore-Penrose pseudoinverse} of $A$.
\section{Principal Component Regression}
% The Principal Component Regression (PCR) is a regression analysis technique that is based on the SVD of the data matrix $A$.
% The PCR is used to overcome the problem of multicollinearity in regression analysis. Multicollinearity occurs when independent variables in a regression model are correlated. This correlation is a problem because independent variables should be independent. If the degree of correlation between variables is high enough, it can cause problems when you fit the model and interpret the results.
In high-dimensional data analysis, it may occur that many of the dimensions of the data are highly correlated.
In these cases, the matrix $A$ is rank-deficient, meaning that its rank is significantly less than the dimension $p$ of
the space it maps to. To address the least squares problem under these conditions,
we employ the singular value decomposition (SVD). The SVD allows us to express $A$ in terms of its singular values and corresponding singular vectors. However, because the solution to the least squares problem is not unique when $A$ is rank-deficient, we seek the solution with the smallest norm.
% When we encounter a matrix $A$ of rank $r$ that is significantly less than the dimension $p$ of the space it maps to, we face a least squares problem. This situation is typical in high-dimensional data analysis where the number of variables (columns of $A$) exceeds the number of observations (rows of $A$).
% To address the least squares problem under these conditions, we employ the singular value decomposition (SVD). The SVD allows us to express $A$ in terms of its singular values and corresponding singular vectors. However, because the solution to the least squares problem is not unique when $A$ is rank-deficient, we seek the solution with the smallest norm.

The matrix $A$ can be decomposed as follows:
\[
A = U \Sigma V^T =
\begin{pmatrix} U_1 & U_2 \end{pmatrix}
\begin{pmatrix}
\Sigma_1 & 0 \\
0 & 0 
\end{pmatrix} \begin{pmatrix}
V_1^T \\
V_2^T
\end{pmatrix}
\]
where $\Sigma_1$ is a diagonal matrix of the non-zero singular values $\sigma_1, \ldots, \sigma_r$, and $\Sigma$ is a square, non-singular matrix of size $r \times r$.
and $U$ and $V$ are the orthogonal matrices containing the left and right singular vectors corresponding to the non-zero singular values.

Using the SVD of $A$, we reframe this problem as:
\[
\begin{aligned}
    \min_x \| Ax - b \|_2^2 &=
    \left\|
        \begin{pmatrix} U_1 & U_2 \end{pmatrix}
        \begin{pmatrix} \Sigma_1 & 0 \\ 0 & 0 \end{pmatrix}
        \begin{pmatrix} V_1^T \\ V_2^T \end{pmatrix}
        x - b
    \right\|_2^2 \\
    & = \left\|
        U \begin{pmatrix}
        \Sigma_1 & 0 \\
        0 & 0 
        \end{pmatrix}
        \begin{pmatrix}
            V_1^T \\
            V_2^T
        \end{pmatrix}
        x - UU^T b
    \right\|_2^2 \\
    & = \left\|
    \begin{pmatrix}
        \Sigma_1 & 0 \\
        0 & 0 
        \end{pmatrix}
        \begin{pmatrix}
            V_1^T \\
            V_2^T
        \end{pmatrix}
        x - 
        \begin{pmatrix}
            U_1^T \\
            U_2^T
        \end{pmatrix}b
    \right\|_2^2 \\
    & = \left\|
        \begin{pmatrix}
            \Sigma_1 V_1^T x - U_1^T b\\
            - U_2^T b
        \end{pmatrix}
    \right\|_2^2 \\
\end{aligned}
\]

\subsection*{Solution to the Least Squares Problem}
To minimize the above expression, we set $y = V^T x$ and partition it as $y_1$ and $y_2$, corresponding to $V_1^T x$ and $V_2^T x$, respectively. The optimal solution is found by minimizing each part of the squared norm separately.

The first part of the squared norm can be minimized to zero when:
\[
\Sigma_1 y_1 = U_1^T b \implies y_1 = \Sigma_1^{-1} U_1^T b
\]
The second part of the squared norm, $\| U_2^T b \|_2^2$, cannot be affected by our choice of $x$ since it depends only on $b$. 

Therefore, to minimize the overall norm $\| x \|_2$, which is equal to $\| y \|_2$, we set $y_2 = 0$ and reconstruct $x$ from $y$:
\[ x = Vy = V\begin{pmatrix} \Sigma_1^{-1} U_1^T b \\ 0 \end{pmatrix} = \begin{pmatrix} V_1 & V_2 \end{pmatrix} \begin{pmatrix} \Sigma_1^{-1} U_1^T b \\ 0 \end{pmatrix} = V_1 \Sigma_1^{-1} U_1^T b \]
\[ x = V \begin{pmatrix} \Sigma_1^{-1} & 0 \\ 0 & 0 \end{pmatrix} U^T b \]

This effectively discards the components of $x$ that would be multiplied by zero in the matrix $A$, as they do not contribute to the solution.

\subsection*{Moore-Penrose Pseudoinverse}
The expression for $x$ we have found is, in fact, the product of $b$ with the Moore-Penrose pseudoinverse of $A$:
\[
x = A^+ b = V \begin{pmatrix} \Sigma_1^{-1} & 0 \\ 0 & 0 \end{pmatrix} U^T b
\]
The pseudoinverse provides a best-fit solution to an overdetermined or underdetermined linear system and is widely used in regression analysis, among other areas.

\section{Relation with Eigenvalues}

Let $ A \in \mathbb{R}^{m \times n} $ with $ m \geq n $ and rank$(A) = r$. We can show the relation among the singular values of $A$ and the eigenvalues of $A^TA$ and $AA^T$.

The singular value decomposition (SVD) of $A$ is given by
\[ A = U \begin{pmatrix}
\Sigma_r & 0 \\
0 & 0 
\end{pmatrix} V^T \]
where $ U \in \mathbb{R}^{m \times m} $, $\Sigma_r$ is a diagonal matrix with the $r$ non-zero singular values, and $ V \in \mathbb{R}^{n \times n} $.

Now, considering $A^TA$:
\[ 
    \begin{aligned}
        A^TA &= V \begin{pmatrix} \Sigma_1 & 0 \\ 0 & 0 \end{pmatrix}^T U^T U\begin{pmatrix} \Sigma_1 & 0 \\ 0 & 0 \end{pmatrix} V^T = V \begin{pmatrix} \Sigma_1 & 0 \\ 0 & 0 \end{pmatrix} \begin{pmatrix} \Sigma_1 & 0 \\ 0 & 0 \end{pmatrix} V^T\\
        &= V \begin{pmatrix} \Sigma_1^2 & 0 \\ 0 & 0 \end{pmatrix}V^T = V\Lambda V^T \quad \Lambda = diag(\sigma_1^2, \ldots, \sigma_r^2, 0, \ldots, 0)
    \end{aligned}
\]
We get $A^TA$ is similar to $\Lambda$, $(A^TA = V\Lambda V^T = V \Lambda V^{-1})$ and
thus has the same eigenvalues $\sigma_1^2, \sigma_2^2, \ldots, \sigma_r^2, 0, \ldots, 0$ and the eigenvectors are $v_1, v_2, \ldots, v_n$.

$$A^TAV = V\Lambda$$
$$
A^TA \begin{pmatrix} v_1 & v_2 & \ldots & v_n \end{pmatrix}
= \begin{pmatrix} v_1 & v_2 & \ldots & v_n \end{pmatrix}
\begin{pmatrix}
    \sigma_1^2 \\
    & \ddots \\
    & & \sigma_r^2 \\
    & & & 0 \\
    & & & & \ddots \\
    & & & & & 0 \\
\end{pmatrix}
$$
$$
\begin{aligned}
    A^TAv_i &= \sigma_i^2 v_i \quad i=1, \ldots, r \\
    A^TAv_i &= 0 \quad i=r+1, \ldots, n
\end{aligned}
$$
Similarly, for $AA^T$:
\[ 
    \begin{aligned}
        AA^T &= U\begin{pmatrix} \Sigma_1 & 0 \\ 0 & 0 \end{pmatrix} V^T V \begin{pmatrix} \Sigma_1 & 0 \\ 0 & 0 \end{pmatrix}^T U^T = U \begin{pmatrix} \Sigma_1 & 0 \\ 0 & 0 \end{pmatrix} \begin{pmatrix} \Sigma_1 & 0 \\ 0 & 0 \end{pmatrix} U^T\\
        &= U \begin{pmatrix} \Sigma_1^2 & 0 \\ 0 & 0 \end{pmatrix} U^T = U\Lambda U^T \quad \Lambda = diag(\sigma_1^2, \ldots, \sigma_r^2, 0, \ldots, 0)
    \end{aligned}
\]

$$AA^TU = U\Lambda$$
$$
AA^T \begin{pmatrix} u_1 & u_2 & \ldots & u_m \end{pmatrix}
= \begin{pmatrix} u_1 & u_2 & \ldots & u_m \end{pmatrix}
\begin{pmatrix}
    \sigma_1^2 \\
    & \ddots \\
    & & \sigma_r^2 \\
    & & & 0 \\
    & & & & \ddots \\
    & & & & & 0 \\
\end{pmatrix}
$$
$$
\begin{aligned}
    AA^Tu_i &= \sigma_i^2 u_i \quad i=1, \ldots, r \\
    AA^Tu_i &= 0 \quad i=r+1, \ldots, m
\end{aligned}
$$

\end{document}