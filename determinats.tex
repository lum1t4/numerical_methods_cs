\chapter*{Determinats}

\begin{enumerate}
    \item Let $A$ be a square matrix 2x2, than its determinat can be easily computed.
        $$
        A = \begin{pmatrix}
            a & b \\
            c & d
        \end{pmatrix}, \quad det(A) := ad - bc
        $$
    \item $det(I) = 1$
    \item The sign of the determinat is changed at every row (or column) permutation.
          If there is an even number of permutations, the sign is positive, otherwise the sign is negative
    \item The determinat is liean in every row every row separately (as well as in every column separately).
    Let's sum the first row of A with another vector:
    $$
    \begin{aligned}
        \tilde{A} &= \begin{pmatrix}
            a + a' & b + b' \\
            c & d
        \end{pmatrix} \\
        det(\tilde{A}) &= (a + a')(d) - (b + b')c \\
        &= ad + a'd+ - bc + b'c \\
        &= ad -bc + a'd - b'c \\
        &=  det(A) + det\begin{pmatrix}
            a' & b' \\
            c & d
        \end{pmatrix}
    \end{aligned}
    $$
    \item IF two rows (or columns) of A are equal, then $det(A) = 0$
    \item Subtracting a multiple of one row (or column) from another row (or column) does not change the determinat.
    \item If there is a row consisting of all zeros, then $det(A) = 0$
    \item If A is triangular, then the determinat is the product of the elements of the main diagonal.
          $$ det(A) = \prod_{i=1}^n a_{ii} \quad \text{with } A \quad n \times n $$
    \item If A is singular, then $det(A) = 0$. If A is invertible (or not singular), then $det(A) \neq 0$
    \item (Binet Theorem). If the determinat of the product of two matrices is equal to the product of the determinats of the matrices
          $$ det(AB) = det(A)det(B)$$
    \item $det(A^-1) = \frac{1}{det(A)}$
          $$ det(A^-1A) = det(A^-1)det(A) = det(I) = 1 \Rightarrow det(A^-1) = \frac{1}{det(A)}$$
    \item $det(A^T) = det(A)$
    \item $det\begin{pmatrix}
        A & 0 \\
        C & D\end{pmatrix} = det(A) \cdot det(D)$ with $A$ and $D$ square matrices
    \item If the given matrxi A is nxn with n > 2, in general we can use the LU factorization because it is a numerically stable method.
          $$
          \begin{aligned}
            PAQ &= LU \Rightarrow det(PAQ) = det(LU) \\
            & \Rightarrow det(P)det(A)det(Q) = det(L)det(U), \quad det(L) = 1, det(P) = \pm 1, det(Q) = \pm 1  \\
            &\Rightarrow det(A) = \pm det(U) = \pm \prod_{i = 1}^{n} u_{ii}
          \end{aligned}
          $$
    \item It is not true that small determinat implies near singularity.
          Example:
          $$ A = \begin{pmatrix}
              \alpha & & O \\
                & \ddots & \\
                O & & \alpha
            \end{pmatrix} \quad det(A) = \alpha^n $$
            If $\alpha$ is small, then $det(A)$ is small, but A is not singular.
    \item It is not true that if the determinat is different from 0, then the matrix is "nice" to invert.
          Example: 
            $$ A = \begin{pmatrix}
                n & 0 \\
                0 & \frac{1}{n}
            \end{pmatrix}, \quad det(A) = 1 $$
            The matrix is badbly scaled!
\end{enumerate}

% TODO: ex 6.1.3 from Mayer Book
