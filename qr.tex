\chapter{QR factorization}
QR factorization is a decomposition of a matrix into a product of an orthogonal matrix and an upper triangular matrix.
It is commonly used in numerical algorithms, particularly for solving systems of linear equations and eigenvalue problems.

Given a square matrix $A \in \mathbb{R}^{n \times n}$, the QR factorization is a decomposition of $A$ into:
\begin{equation*}
    A = QR
\end{equation*}
where $Q$ is an $m \times m$ orthogonal matrix ($Q^TQ = QQ^T = I$) and $R$ is an $m \times n$ upper triangular matrix.

The factorization is can be computed using the Gram-Schmidt process, Householder reflections, or Givens rotations.
Householder reflections are more numerically stable and thus commonly used in practice.
Givens rotations are also stable but generally require more operations so this method is computationaly costly
and so are used primarily for sparse matrices where less operations are required to zero out the below-diagonal elements.
% using the Golub–Reinsch algorithm.

To perform QR factorization, we aim to construct an orthogonal matrix $Q$ such that $Q^TA$ is upper triangular. We start by finding a Householder reflector that zeroes out the below-diagonal elements of the first column of $A$.

Let $A_{*1}$ be the first column of $A$, and $e_1$ be the first unit vector of the same size. We choose a vector $u$ such that:
\begin{equation*}
    u = A_{*1} - k_1e_1 \quad k_1 = \pm \|A_{*1}\|_2
\end{equation*}

\begin{mdframed}
How do we choose the sign of $k_1$ (similarly $k_j$)?
We want to minimize the roundoff errors by avoiding subtraction
(catastrophic cancellation) so
\[
k_1 = \begin{cases}
    -\|A_{*1}\|_2 & \text{if } A_{11} \geq 0 \\
    \|A_{*1}\|_2 & \text{if } A_{11} < 0
\end{cases}
\]
\end{mdframed}

Then the reflector $Q_1$ is:
\begin{equation*}
    Q_1 = I - 2\frac{uu^T}{u^Tu}
\end{equation*}
Applying $Q_1$ to $A$ gives us a matrix which has zeroes below the diagonal in the first column:
\begin{equation*}
    A_1 = Q_1A
\end{equation*}
This process is repeated for the subsequent columns of $A$,
but we apply the Householder reflectors to the supmatrix $A_i$ that
excludes the rows and columns that have already been processed.
This can be achived "padding" the computed reflector with an identity matrix:
$$ Q_i = \begin{pmatrix}
    I_{i-1} & 0 \\
    0 & R_i
\end{pmatrix} \quad  R_iA_i = \begin{pmatrix}
    k_i & * & \cdots & * \\
    0 & * & \cdots & * \\
    \vdots & \vdots & \ddots & \vdots \\
    0 & * & \cdots & *
\end{pmatrix}
$$
After applying the necessary Householder reflectors, we obtain:
\begin{equation*}
    Q = Q_1Q_2...Q_n
\end{equation*}
\begin{equation*}
    R = Q^TA
\end{equation*}
The matrices $Q$ and $R$ constitute the QR factorization of $A$.

\section{QR Factorization for rectangular matrices}
Let $A \in \mathbb{R}^{m \times n}$. If we consider the case $m < n$.
\[
A = \left.
\left( \begin{array}{c}
\underbrace{\rule{3cm}{0pt}}_{n} \\
\end{array} \right)
\right\}m
\longrightarrow Q_{m-1}\ldots Q_1A = \begin{pmatrix}
    * & * & \cdots & *  & * & * \\
    0 & * & \cdots & *  & * & * \\
    \vdots & \ddots & \ddots & \vdots & \ddots & \vdots \\
    0 & \cdots & 0 & * & \cdots & * 
\end{pmatrix} = T
\]

T is an upper trapezoidal matrix $ T = \begin{pmatrix} R & * \end{pmatrix}$.

We consider the case $m > n$.
\[
A = \left.
\begin{pmatrix}
    \underbrace{\rule{0pt}{3cm}}_{n} \\
\end{pmatrix}
\right\}m
\longrightarrow Q_{n}\ldots Q_1A = \begin{pmatrix}
    * & 0 & \cdots & 0 \\
    * & * & \cdots & 0 \\
    \vdots & \vdots & \ddots & \vdots \\
    * & * & \cdots & *
\end{pmatrix} = T = \begin{pmatrix}
    R \\ O
\end{pmatrix}
\]

$$
Q = \begin{pmatrix} Q_n \cdot Q_{n-1} \cdot \ldots \cdot Q_1 \end{pmatrix}^T \quad
\underbrace{A}_{m \times n} = \underbrace{U}_{m \times m} \underbrace{T}_{m \times n}
= \begin{pmatrix}
    \underbrace{U_1}_{n} & \underbrace{U_2}_{m-n}
\end{pmatrix}
\begin{pmatrix}
    R \\ O
\end{pmatrix}
= Q_1R
$$

Not all the columns of $Q$ are important, $Q_1$ hase all information

Let us suppose that the columns of $A$ are linearly independent and we wish to compute an orthogonal basis.
For span $\{A_1, A_2, \ldots, A_n\}$ we use the GS algorithm we compute $u_1, u_2, \ldots, u_n$
\[
A = \begin{pmatrix}
    u_1 & u_2 & \cdots & u_n
\end{pmatrix}
\begin{pmatrix}
    r_{11} & r_{12} & \cdots & r_{1n} \\
    & r_{22} & \cdots & r_{2n} \\
    & & \ddots & \vdots \\
    & & & r_{nn}
\end{pmatrix}
\]

we called the thin QR factorization

$U_1$ is not an orthogonal matrix since $U_1^T U_1 = I$ but $U_1 U_1^T \neq I$.


\subsection{Orthogonal Reduction}
For every matrix $A \in \mathbb{C}^{m \times n} (\mathbb{R}^{m \times n})$ there exists a unitary matrix $U$
(orthogonal in $\mathbb{R}$) and an upper trapezoidal matrix $T$ such that
$$A = UT$$

\begin{equation*}
S = \begin{pmatrix}
1 - i\sqrt{2} & 2+i\sqrt{2} \\
2 - i3 & 3+i4
\end{pmatrix}
\end{equation*}
\begin{equation*}
S \in \mathbb{C}^{2 \times 2}
\end{equation*}
\begin{equation*}
\bar{S} = \begin{pmatrix}
1+i\sqrt{2} & 2-i\sqrt{2} \\
2+i3 & 3-i4
\end{pmatrix}
\end{equation*}
\begin{equation*}
\bar{S}^T = S^* = \begin{pmatrix}
1+i\sqrt{2} & 2+i3 \\
2-i\sqrt{2} & 3-i4
\end{pmatrix}
\end{equation*}

If $A$ is square, $T$ is upper triangular. This factorization is denoted as the QR factorization. We can compute the QR factorization using:
\begin{enumerate}
\item the Householder transformation
\item Gram-Schmidt algorithm (not numerically stable)
\item Modified Gram-Schmidt algorithm (numerically stable)
\item The Givens Rotation (elementary orthogonal matrices)
    They are useful when \( A \) is sparse (has a lot of entries equal to zero).
\end{enumerate}

\section{Pivoted QR Decomposition}
(Rank revealing decomposition)

\[
A_{m \times n} P_{n \times n} = _{m \times n}
\]

\[
t_{11}, t_{22}, \ldots, t_{pp} \quad p = \min(m, n)
\]

We choose the permutation matrix \( P \) in order to have in \( QR \):

\[
|t_{11}| \geq |t_{22}| \geq \cdots \geq |t_{pp}|
\]

The rank of the matrix:

\[
|t_{rr}| \geq tol = \kappa\varepsilon
\]


\[
\text{If } \text{tr}(A) \geq 0 \text{ means the rank is } r
\]

If we use the Rank Revealing QR

\[
A_{m \times n} P_{n \times n} = U_{m \times m} \begin{pmatrix}
T_{11} & T_{12} \\
0 & 0
\end{pmatrix}_{m \times n} = U T
\]

\[
\begin{aligned}
&T_{11} = \begin{pmatrix}
\ast & \ast \\
0 & \ast
\end{pmatrix} \text{ of size } r \times r \\
&T_{12} = \begin{pmatrix}
\ast & \ast \\
\ast & \ast
\end{pmatrix} \text{ a full matrix} \text{ of size } r \times (n-r)
\end{aligned}
\]

\section{Numerical Stability of the QR Factorization}





An algorithm is numerically stable if when performed using floating point arithmetic it gives a solution close to the solution computed with exact arithmetic.

Let $A \in \mathbb{R}^{n \times m}$, we compute the QR factorization in exact and floating point arithmetic

\begin{align*}
A &= QR \quad \\
\hat{A} &= (Q+E)(R+F) = QR + QF + ER + EF
\end{align*}

If the Householder or Givens matrix $E$ and $F$ contains the floating point errors
\begin{align*}
\|E\| &= \kappa_u \\
\|F\| &= \kappa_u \\
\|EF\| &\approx \kappa_u^2 \quad u \ll \kappa_u
\end{align*}

we neglect $u^2$.

\section*{Matrix Approximation}

Given a matrix approximation:
\[
\hat{A} \approx QR + QF^T + \epsilon R
\]
We can also express it as:
\[
\hat{A} \times A + QF^T + \epsilon R
\]

\section*{Norms and Orthogonality}

Considering the orthogonality of $Q$:
\[
\|\hat{A}Q\|_2 = \|AQ\|_2 \quad \text{because $Q$ is orthogonal}
\]
We have the following inequalities:
\[
\|EQ\|_2 \leq \|E\|_2\|Q\|_2
\]
And equivalences:
\[
\|AQ\|_2 = \|QRQ\|_2 = \|RQ\|_2
\]
Further inequalities:
\[
\|\hat{A}Q\|_2 \leq \|QR\|_2 + \|QF\|_2 + \|EQ\|_2\|RQ\|_2
\]
\[
\|\hat{A}Q\|_2 \leq (1 + \|EQ\|_2)\|RQ\|_2 + \|QF\|_2
\]
And finally:
\[
\|\hat{A} - AQ\|_2 \leq \|QF\|_2 + \|EQ\|_2\|RQ\|_2
\]

\section*{Conclusion}

The approximation $\hat{A}$ is very close to $A$. The factorization is stable. We do the same for the LU factorization.

\section*{Mathematical Expressions}

\begin{align*}
A &= LU \\
A &= (L + E)(U + F) \\
A &= LU + LF + EU + EF \\
A &\approx LU + LF + EU
\end{align*}

\begin{mdframed}[backgroundcolor=blue!20]
\[\|LF\| \leq (\underbrace{\|L\|}_{\text{could be very big}}) (\|F\|)\]
\end{mdframed}

\begin{mdframed}[backgroundcolor=blue!20]
\[\|EU\| \leq \|E\|(\underbrace{\|U\|}_{\text{could be very big}})\]
\end{mdframed}

\section*{Pivoting Strategy}

If we use the pivoting $\|L\|$ is bounded and $\|LF\| \leq k\|F\|$ and $\|U\| \approx \|A\|$

\section*{Computational Effort}

\begin{itemize}
\item LU with partial pivoting $\approx n^3/3$
\item Householder $\approx 2n^3/3$
\end{itemize}

\section{Linear Regression}





We have a set of variables $t_1, t_2, \ldots, t_n$ and we would like to express another variable $y$ as a linear combination of $t_1, t_2, \ldots, t_n$:
\[
y = \alpha_0 + \alpha_1 t_1 + \alpha_2 t_2 + \ldots + \alpha_n t_n + \varepsilon
\]
$\varepsilon$ is a random function with mean zero:
\[
\mathbb{E}(y) = \alpha_0 + \alpha_1 t_1 + \alpha_2 t_2 + \ldots + \alpha_n t_n
\]
(expected value).

We know
\[
\begin{pmatrix}
t_1^{(i)} & t_2^{(i)} & \ldots & t_n^{(i)}
\end{pmatrix}
\]
and $y^{(i)}$ are i.i.d. $\sim \mathcal{N}$

\begin{mdframed}[backgroundcolor=blue!20]
Exercise 4.6.8 of the book
Solve it with the QR factorization using \textsc{Python}
\end{mdframed}
