\chapter{URV Factorization}
\textbf{Theorem:} Given $A \in \mathbb{R}^{m \times n}$ a matrix with $\text{rank}(A)=r$, there exist two orthogonal matrices $U \in \mathbb{R}^{m \times m}$ and $V \in \mathbb{R}^{n \times n}$ and a nonsingular matrix $C \in \mathbb{R}^{r \times r}$ such that
\[
A =
\underbrace{U}_{m \times m}
\underbrace{
    \begin{pmatrix}
    C & 0 \\
    0 & 0
    \end{pmatrix}
}_{m \times n}
\underbrace{V^T}_{n \times n} = URV^T
\quad
\text{where}
\quad
R = \begin{pmatrix}
C_{r \times r} & 0 \\
0 & 0
\end{pmatrix}
\]
This factorization is not unique.

\section{Properties of URV}
\begin{enumerate}
\item The first $r$ columns of $U$ are an orthonormal basis for $\mathcal{R}(A)$.
\item The first $r$ columns of $V$ are an orthonormal basis for $R(A^T)$
\item The last $m-r$ columns of $U$ are an orthonormal basis for the $N(A^T)$
\item The last $n-r$ columns of $V$ are an orthonormal basis for the $N(A)$
\end{enumerate}

Suppose that we know orthonormal basis for the four fundamental subspaces
$R(A)$, $N(A)$ and $R(A^T)$, $N(A^T)$.
We construct the matrix $U$ and $V$ as follows:
$$U := \underbrace{\left(u_1, u_2, \cdots, u_r\right.}_{\text{basis for } R(A)} \underbrace{\left.u_{r+1}, \cdots, u_m\right)}_{\text{basis for } N(A^T)}$$
$$V := \underbrace{\left(v_1, v_2, \cdots, v_r\right.}_{\text{basis for } R(A^T)} \underbrace{\left.v_{r+1}, \cdots, v_n\right)}_{\text{basis for } N(A)}$$

\[
U^T A V = U^T \left( U \begin{pmatrix}
C & 0 \\
0 & 0
\end{pmatrix} V^T \right) V = \begin{pmatrix}
C & 0 \\
0 & 0
\end{pmatrix} = R
\]

\[
\begin{aligned}
    R = U^T A V &= \begin{pmatrix}
        u_1^T \\
        \cdots \\
        u_r^T \\
        u_{r+1}^T \\
        \cdots \\
        u_m^T
        \end{pmatrix} A \begin{pmatrix}
        v_1 & \cdots & v_r & v_{r+1} & \cdots & v_n
        \end{pmatrix} \\
    &= \begin{pmatrix}
            u_1^T \\
            \cdots \\
            u_r^T \\
            u_{r+1}^T \\
            \cdots \\
            u_m^T
        \end{pmatrix}
        \begin{pmatrix}
            Av_1 & \cdots & Av_r & Av_{r+1} & \cdots & Av_n
        \end{pmatrix} \\
    &= \begin{pmatrix}
        u_1^T A v_1 & \cdots & u_1^T A v_r & u_1^T A v_{r+1} & \cdots & u_1^T A v_n \\
        \vdots & \ddots & \vdots & \vdots & & \vdots \\
        u_r^T A v_1 & \cdots & u_r^T A v_r & u_r^T A v_{r+1} & \cdots & u_r^T A v_n \\
        u_{r+1}^T A v_1 & \cdots & u_{r+1}^T A v_r & u_{r+1}^T A v_{r+1} & \cdots & u_{r+1}^T A v_n \\
        \vdots & & \vdots & \vdots & \ddots & \vdots \\
        u_m^T A v_1 & \cdots & u_m^T A v_r & u_m^T A v_{r+1} & \cdots & u_m^T A v_n
        \end{pmatrix}
\end{aligned}
\]

We note that the vectors \( u_{r+1}, \ldots, u_m \) are in the null space of \( A^T \), which leads to:
\[
\begin{aligned}
&u_{r+1}^T \in N(A^T) \implies A^T u_{r+1} = 0 \implies u_{r+1}^T A = 0 \implies &u_{r+1}^T A v_j = 0 \\
\vdots \\
&u_m^T \in N(A^T) \implies A^T u_m = 0 \implies u_m^T A = 0 \implies &u_m^T A v_j = 0 
\end{aligned}
\]
$$ \Rightarrow u_i^TAv_j = 0 \quad \forall i = r+1, \dots, m \quad \forall j = 1, \dots, n $$

and the vectors \( v_{r+1}, \ldots, v_n \) are in the null space of \( A \), which leads to:
\[
\begin{aligned}
&v_{r+1}^T \in N(A) \implies Av_{r+1} = 0 \implies u_i^T A v_{r+1} = 0 \\
\vdots \\
&v_n^T \in N(A) \implies Av_n = 0 \implies u_i^T A v_n = 0
\end{aligned}
\]
$$ \Rightarrow u_i^TAv_j = 0 \quad \forall i = 1, \dots, m \quad \forall j = r+1, \dots, n $$

So the matrix $R$ is defined as:
\[ R_{ij} = u_i^T A u_j \neq 0 \quad \text{for} \quad i \leq r, \quad j \leq r \]

\[
R = \begin{pmatrix}
u_1^T A v_1 & \cdots & u_1^T A v_r & 0 & \cdots & 0 \\
\vdots & \ddots & \vdots & \vdots & \ddots & \vdots \\
u_r^T A v_1 & \cdots & u_r^T A v_r & 0 & \cdots & 0 \\
0 & \cdots & 0 & 0 & \cdots & 0 \\
\vdots & \ddots & \vdots & \vdots & \ddots & \vdots \\
0 & \cdots & 0 & 0 & \cdots & 0
\end{pmatrix}
= \begin{pmatrix} C & 0 \\ 0 & 0 \end{pmatrix}
\]


\section{Computing the URV Factorization using the QR Factorization}
Using the QR Factorization we have:
\[ A P = Q \begin{pmatrix} T \\ 0 \end{pmatrix} \]
% \[ A_{m \times n} P_{n \times n} = Q_{m \times m} \begin{pmatrix} T_{r \times n} \\ 0 \end{pmatrix}_{m \times n} \]
where
\begin{itemize}
    \item $A$ is a matrix $m \times n$
    \item $P$ is a permutation matrix $n \times n$
    \item $Q$ is an orthogonal matrix $m \times m$
    \item $T$ is an upper trapezoidal matrix $r \times n$
    \item $\begin{pmatrix} T \\ 0 \end{pmatrix}$ is $m \times n$
\end{itemize}

$$ U = Q \implies AP = U \begin{pmatrix} T \\ 0 \end{pmatrix} $$
The range of $T^T$ and the range of $A^T$ are equal. \newline
Let \( r < \min(m, n) \).
\[ T^T = Q \begin{pmatrix} B \\ 0 \end{pmatrix} \quad B \text{ is an upper triangular matrix } r \times r \]
\[ T = \left(Q\begin{pmatrix} B \\ O \end{pmatrix}\right)^T = \begin{pmatrix} B^T & O \end{pmatrix} \cdot Q^T \]
\[ A P = U \begin{pmatrix} \begin{pmatrix} B^T & O \end{pmatrix} Q^T \\ \begin{pmatrix} O & O \end{pmatrix} Q^T \end{pmatrix} = U \begin{pmatrix} B^T & O \\ O & O \end{pmatrix} Q^T \]
\[ A = U \begin{pmatrix} C & O \\ O & O \end{pmatrix} Q^T P^T = U \begin{pmatrix} C & 0 \\ 0 & 0 \end{pmatrix}V^T \]
\[ V^T = Q^T P^T, \quad C = R^T = \begin{pmatrix}
\ast & 0 & \cdots & 0 \\
\ast & \ast & \cdots & 0 \\
\vdots & \vdots & \ddots & \vdots \\
\ast & \ast & \cdots & \ast 
\end{pmatrix}
\]

$r_{11}, r_{22}, \ldots, r_{rr}$ are sorted $|r_{11}| \geq |r_{22}| \geq \ldots \geq |r_{rr}|$.

The most simple matrix $C$ is with with a diagonal structure
\[
C = D = \begin{pmatrix}
    \ast & 0 & \cdots & 0 \\
    0 & \ast & \cdots & 0 \\
    \vdots & \vdots & \ddots & \vdots \\
    0 & 0 & \cdots & \ast 
    \end{pmatrix}
\]

\begin{tabular}{l l}
BELTRAMI & 1873 \\
JORDAN & 1875 \\
SYLVESTER & 1893 \\
ECKART-YOUNG & 1936 \\
\end{tabular}