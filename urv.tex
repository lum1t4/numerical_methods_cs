\chapter{URV Factorization}
\textbf{Theorem:} Given $A \in \mathbb{R}^{m \times n}$ a matrix with $\text{rank}(A)=r$, there exist two orthogonal matrices $U \in \mathbb{R}^{m \times m}$ and $V \in \mathbb{R}^{n \times n}$ and a nonsingular matrix $C \in \mathbb{R}^{r \times r}$ such that
\[
A =
\underbrace{U}_{m \times m}
\underbrace{
    \begin{pmatrix}
    C & 0 \\
    0 & 0
    \end{pmatrix}
}_{m \times n}
\underbrace{V^T}_{n \times n} = URV^T
\quad
\text{where}
\quad
R = \begin{pmatrix}
C_{r \times r} & 0 \\
0 & 0
\end{pmatrix}
\]
This factorization is not unique.

\section{Properties of URV}
\begin{enumerate}
\item The first $r$ columns of $U$ are an orthonormal basis for $\mathcal{R}(A)$.
\item The first $r$ columns of $V$ are an orthonormal basis for $R(A^T)$
\item The last $m-r$ columns of $U$ are an orthonormal basis for the $N(A^T)$
\item The last $n-r$ columns of $V$ are an orthonormal basis for the $N(A)$
\end{enumerate}

Suppose that we know orthonormal basis for the four fundamental subspaces
$R(A)$, $N(A)$ and $R(A^T)$, $N(A^T)$.
We construct the matrix $U$ and $V$ as follows:
$$U := \underbrace{\left(u_1, u_2, \cdots, u_r\right.}_{\text{basis for } R(A)} \underbrace{\left.u_{r+1}, \cdots, u_m\right)}_{\text{basis for } N(A^T)}$$
$$V := \underbrace{\left(v_1, v_2, \cdots, v_r\right.}_{\text{basis for } R(A^T)} \underbrace{\left.v_{r+1}, \cdots, v_n\right)}_{\text{basis for } N(A)}$$

\[
U^T A V = U^T \left( U \begin{pmatrix}
C & 0 \\
0 & 0
\end{pmatrix} V^T \right) V = \begin{pmatrix}
C & 0 \\
0 & 0
\end{pmatrix} = R
\]

\[
\begin{aligned}
    R = U^T A V &= \begin{pmatrix}
        u_1^T \\
        \cdots \\
        u_r^T \\
        u_{r+1}^T \\
        \cdots \\
        u_m^T
        \end{pmatrix} A \begin{pmatrix}
        v_1 & \cdots & v_r & v_{r+1} & \cdots & v_n
        \end{pmatrix} \\
    &= \begin{pmatrix}
            u_1^T \\
            \cdots \\
            u_r^T \\
            u_{r+1}^T \\
            \cdots \\
            u_m^T
        \end{pmatrix}
        \begin{pmatrix}
            Av_1 & \cdots & Av_r & Av_{r+1} & \cdots & Av_n
        \end{pmatrix} \\
    &= \begin{pmatrix}
        u_1^T A v_1 & \cdots & u_1^T A v_r & u_1^T A v_{r+1} & \cdots & u_1^T A v_n \\
        \vdots & \ddots & \vdots & \vdots & & \vdots \\
        u_r^T A v_1 & \cdots & u_r^T A v_r & u_r^T A v_{r+1} & \cdots & u_r^T A v_n \\
        u_{r+1}^T A v_1 & \cdots & u_{r+1}^T A v_r & u_{r+1}^T A v_{r+1} & \cdots & u_{r+1}^T A v_n \\
        \vdots & & \vdots & \vdots & \ddots & \vdots \\
        u_m^T A v_1 & \cdots & u_m^T A v_r & u_m^T A v_{r+1} & \cdots & u_m^T A v_n
        \end{pmatrix}
\end{aligned}
\]

We note that the vectors \( u_{r+1}, \ldots, u_m \) are in the null space of \( A^T \), which leads to:
\[
\begin{aligned}
&u_{r+1}^T \in N(A^T) \implies A^T u_{r+1} = 0 \implies u_{r+1}^T A = 0 \implies &u_{r+1}^T A v_j = 0 \\
\vdots \\
&u_m^T \in N(A^T) \implies A^T u_m = 0 \implies u_m^T A = 0 \implies &u_m^T A v_j = 0 
\end{aligned}
\]
$$ \Rightarrow u_i^TAv_j = 0 \quad \forall i = r+1, \dots, m \quad \forall j = 1, \dots, n $$

and the vectors \( v_{r+1}, \ldots, v_n \) are in the null space of \( A \), which leads to:
\[
\begin{aligned}
&v_{r+1}^T \in N(A) \implies Av_{r+1} = 0 \implies u_i^T A v_{r+1} = 0 \\
\vdots \\
&v_n^T \in N(A) \implies Av_n = 0 \implies u_i^T A v_n = 0
\end{aligned}
\]
$$ \Rightarrow u_i^TAv_j = 0 \quad \forall i = 1, \dots, m \quad \forall j = r+1, \dots, n $$

So the matrix $R$ is defined as:
\[ R_{ij} = u_i^T A u_j \neq 0 \quad \text{for} \quad i \leq r, \quad j \leq r \]

\[
R = \begin{pmatrix}
u_1^T A v_1 & \cdots & u_1^T A v_r & 0 & \cdots & 0 \\
\vdots & \ddots & \vdots & \vdots & \ddots & \vdots \\
u_r^T A v_1 & \cdots & u_r^T A v_r & 0 & \cdots & 0 \\
0 & \cdots & 0 & 0 & \cdots & 0 \\
\vdots & \ddots & \vdots & \vdots & \ddots & \vdots \\
0 & \cdots & 0 & 0 & \cdots & 0
\end{pmatrix}
= \begin{pmatrix} C & 0 \\ 0 & 0 \end{pmatrix}
\]


\section{Computing the URV Factorization using the QR Factorization}
Using the QR Factorization we have:
\[ A P = Q \begin{pmatrix} T \\ 0 \end{pmatrix} \]
% \[ A_{m \times n} P_{n \times n} = Q_{m \times m} \begin{pmatrix} T_{r \times n} \\ 0 \end{pmatrix}_{m \times n} \]
where
\begin{itemize}
    \item $A$ is a matrix $m \times n$
    \item $P$ is a permutation matrix $n \times n$
    \item $Q$ is an orthogonal matrix $m \times m$
    \item $T$ is an upper trapezoidal matrix $r \times n$
    \item $\begin{pmatrix} T \\ 0 \end{pmatrix}$ is $m \times n$
\end{itemize}

$$ U = Q \implies AP = U \begin{pmatrix} T \\ 0 \end{pmatrix} $$
The range of $T^T$ and the range of $A^T$ are equal. \newline
Let \( r < \min(m, n) \).
\[ T^T = Q \begin{pmatrix} B \\ 0 \end{pmatrix} \quad B \text{ is an upper triangular matrix } r \times r \]
\[ T = \left(Q\begin{pmatrix} B \\ O \end{pmatrix}\right)^T = \begin{pmatrix} B^T & O \end{pmatrix} \cdot Q^T \]
\[ A P = U \begin{pmatrix} \begin{pmatrix} B^T & O \end{pmatrix} Q^T \\ \begin{pmatrix} O & O \end{pmatrix} Q^T \end{pmatrix} = U \begin{pmatrix} B^T & O \\ O & O \end{pmatrix} Q^T \]
\[ A = U \begin{pmatrix} C & O \\ O & O \end{pmatrix} Q^T P^T = U \begin{pmatrix} C & 0 \\ 0 & 0 \end{pmatrix}V^T \]
\[ V^T = Q^T P^T, \quad C = R^T = \begin{pmatrix}
\ast & 0 & \cdots & 0 \\
\ast & \ast & \cdots & 0 \\
\vdots & \vdots & \ddots & \vdots \\
\ast & \ast & \cdots & \ast 
\end{pmatrix}
\]

$r_{11}, r_{22}, \ldots, r_{rr}$ are sorted $|r_{11}| \geq |r_{22}| \geq \ldots \geq |r_{rr}|$.

The most simple matrix $C$ is with with a diagonal structure
\[
C = D = \begin{pmatrix}
    \ast & 0 & \cdots & 0 \\
    0 & \ast & \cdots & 0 \\
    \vdots & \vdots & \ddots & \vdots \\
    0 & 0 & \cdots & \ast 
    \end{pmatrix}
\]

\begin{tabular}{l l}
BELTRAMI & 1873 \\
JORDAN & 1875 \\
SYLVESTER & 1893 \\
ECKART-YOUNG & 1936 \\
\end{tabular}

\section{Singular Value Decomposition}
Singular Value Decomposition (SVD) is a fundamental matrix factorization technique in linear algebra.
The \textbf{SVD is a special case of the URV factorization}, where $C$ is a rectangular diagonal matrix that we denote with $\Sigma$ (Notation of Golub).
For a given matrix $A$ with dimensions $m \times n$, the SVD is a decomposition such that
\[
A = U \Sigma V^T = 
\begin{pmatrix}
\vert & & \vert \\
u_1 & \dots & u_m \\
\vert & & \vert
\end{pmatrix}
\begin{pmatrix}
\sigma_1 & & 0 \\
& \ddots & \\
0 & & \sigma_n
\end{pmatrix}
\begin{pmatrix}
\text{---} & v_1^T & \text{---} \\
& \vdots & \\
\text{---} & v_n^T & \text{---}
\end{pmatrix}
\]
where
\begin{itemize}
    \item $U$ is an $m \times m$ orthogonal matrix containing the left-hand singular vectors of $A$.
    \item $V$ is an $n \times n$ orthogonal matrix containing the right-hand singular vectors of $A$.
    \item $\Sigma$ is an $m \times n$ rectangular diagonal matrix with non-negative real numbers on the diagonal.
\end{itemize}

The diagonal entries $\sigma_i$ of $\Sigma$ are known as the singular values of $A$
and are arranged such that $\sigma_1 \geq \sigma_2 \geq \dots \geq \sigma_r > 0$,
where $r$ is the rank of $A$. \newline
\textbf{The number of non-zero singular values is equal to the rank of $A$.}

\subsection{Properties of SVD}
\begin{itemize}
    \item The columns of $U$ are the eigenvectors of $AA^T$.
    \item The columns of $V$ are the eigenvectors of $A^TA$.
    \item The non-zero singular values of $A$ (the $\sigma_i$'s) are the square roots of the non-zero eigenvalues of both $AA^T$ and $A^TA$.
\end{itemize}

\subsection{Applications of SVD}
SVD is widely used in statistics and signal processing for tasks such as:
\begin{itemize}
    \item Data compression
    \item Noise reduction
    \item Pseudoinverse computation
    \item Statistics (Hotelling transform)
    \subitem Principal Component Analysis (PCA)
    \item Image processing (Karhunen-Loève transform)
\end{itemize}

\subsection{Relation to the Eigenvalue Decomposition}
SVD is a generalization of the eigenvalue decomposition for non-square matrices. In the case where $A$ is square and symmetric, the SVD is equivalent to the eigenvalue decomposition.

If $m > n \quad m=3, n=2$, we have:
\[
\Sigma = \begin{pmatrix}
\sigma_1 & 0 \\
0 & \sigma_2 \\
0 & 0
\end{pmatrix}
\]

If $m < n \quad m=2, n=3$, we have:
\[
\Sigma = \begin{pmatrix}
\sigma_1 & 0 & 0 \\
0 & \sigma_2 & 0
\end{pmatrix}
\]

If $\sigma_2 = 0$, then $\Sigma$ simplifies to:
\[
\Sigma = \begin{pmatrix}
\sigma_1 & 0 & 0 \\
0 & 0 & 0
\end{pmatrix} = \begin{pmatrix}
\sigma_1 & 0 \\
0 & 0
\end{pmatrix}
\]

\section*{Elden Notation}
% TODO: check if useful
In the book of \textit{Elden}, if $m > n$, the matrix $A$ can be decomposed as:
\[
A = U \begin{pmatrix}
\Sigma \\
0
\end{pmatrix} V^T
\]
\[
= U \begin{pmatrix}
\sigma_1 & 0 \\
0 & 0
\end{pmatrix} V^T
\]

For $\Sigma$ given by:
\[
\Sigma = \begin{pmatrix}
\sigma_1 & 0 \\
0 & \sigma_2
\end{pmatrix} \quad \text{if } \sigma_2 = 0, \quad \Sigma = \begin{pmatrix}
\sigma_1 & 0 \\
0 & 0
\end{pmatrix}
\]

If $A = U \begin{pmatrix}
\sigma_1 & 0 \\
0 & 0
\end{pmatrix} V^T$, then only the first $r$ columns of $U$ and the first $r$ columns of $V$ are useful to describe $A$ for $r \times r$.

\subsection{SVD compact form}
$$
A = \underbrace{\left(u_1, u_2, \cdots, u_r\right.}_{U_A} \underbrace{\left.u_{r+1}, \cdots, u_m\right)}_{U_A^\perp}
\begin{pmatrix}
D & 0 \\
0 & 0
\end{pmatrix}
    \begin{pmatrix}
        v_1^T \\
        v_2^T \\
        \vdots \\
        v_r^T \\
        v_{r+1}^T \\
        \vdots \\
        v_n^T
        \end{pmatrix}
    \begin{aligned}
        &\left.
            \vphantom{\begin{matrix}
                u_1 \\
                u_2 \\
                \vdots \\
                u_r 
                \end{matrix}}
        \right\}U_A \\
        &\left.
            \vphantom{\begin{matrix}
                u_1 \\
                \vdots \\
                u_r 
                \end{matrix}}
        \right\}U_A^\perp
    \end{aligned}
$$
    
This can be compactly written as:
\[
A = U_A D V_A^T
\]
which corresponds to the output obtained when setting the parameter "full matrices = false" on available SVD routines.

\subsection{A as a sum of rank 1 matrices}
If the matrix \( A \) has rank \( r = 1 \), then the SVD is given by:
\[
A = u_i \sigma_i v_i^T = \sigma_i \underbrace{u_i v_i^T}_{\text{outer}}
\]
and we have that $\| u_1 \| = 1$, $ \| v_1 \| = 1$, $\| A \|_2 = \sigma_1$.

If the rank \( r > 1 \), then the matrix \( A \) can be represented as:
$$
\begin{aligned}
A &= \begin{pmatrix}
    u_1 & u_2 & \cdots & u_n
    \end{pmatrix}
\begin{pmatrix}
\sigma_1 & 0 & \ldots & 0 \\
0 & \sigma_2 & \ldots & 0 \\
\vdots & \vdots & \ddots & \vdots \\
0 & 0 & \ldots & \sigma_r
\end{pmatrix} \begin{pmatrix}
v_1^T \\
v_2^T \\
\vdots \\
v_r^T
\end{pmatrix} \\
&=
\begin{pmatrix}
u_1 & u_2 & \cdots & u_n
\end{pmatrix}
\begin{pmatrix}
\sigma_1 v_1^T \\
\sigma_2 v_2^T \\
\vdots \\
\sigma_r v_r^T
\end{pmatrix} \\
&=
\sigma_1 u_1 v_1^T + \sigma_2 u_2 v_2^T + \cdots + \sigma_r u_r v_r^T
\end{aligned}
$$

The matrix \( A \) is the sum of rank 1 matrices

\[
A = \sum_{i=1}^{r} \sigma_i u_i v_i^T \quad r \text{ is the rank}
\]

Size \( 1000 \times 1000 \) \quad \( m = n = 1000 \)

\[
\begin{aligned}
\sigma_1 &= 10^5 & \sigma_2 &= 10^4 & \sigma_3 &= 10^0 & \sigma_4 &= 10^0 \\
\sigma_5 &= 10^{-10} & & & & & & \\
& & \sigma_i &\leq 10^{-15} & & & \lambda_i &\leq 10^{100} \\
& & & & & & & \\
& & & & & & \left(10^{200}\right) & \\
\end{aligned}
\]
