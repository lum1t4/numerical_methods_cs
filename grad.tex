\chapter{Derivates}
% From Gilbert Strang book chapter 13
$$ f(x,y) x \in \mathbb{R}, y \in \mathbb{R} $$


Example

$$ \min_{x} \|Ax-b\|^2_2 = \min_x f(x) $$

Up to now, we use linear Algebra to solve this problem.
Now we want to use calculus.



$$
x = \begin{pmatrix}
x_1 \\
x_2
\end{pmatrix}
$$
$$
A = \begin{pmatrix}
a_{11} & a_{12} \\
a_{21} & a_{22}
\end{pmatrix}
$$
$$
b = \begin{pmatrix}
b_1 \\
b_2
\end{pmatrix}
$$

$$
\min_{x_1, x_2} f(x) = \min_{x_1, x_2} \left( (a_{11}x_1 + a_{12}x_2 - b_1)^2 + (a_{21}x_1 + a_{22}x_2 - b_2)^2 \right)
$$


\section{Partial Derivatives}

Partial derivatives are a fundamental concept in calculus, particularly in the field of multivariable calculus. They represent the rate of change of a function with respect to one of its variables, while keeping the other variables constant.

Let \( f: \mathbb{R}^n \rightarrow \mathbb{R} \) be a function of \( n \) variables, say \( f(x_1, x_2, \ldots, x_n) \). The partial derivative of \( f \) with respect to its \( i \)-th variable \( x_i \) is denoted as \( \frac{\partial f}{\partial x_i} \) and is defined as:

\[
\frac{\partial f}{\partial x_i}(x_1, x_2, \ldots, x_n) = \lim_{h \rightarrow 0} \frac{f(x_1, \ldots, x_i + h, \ldots, x_n) - f(x_1, \ldots, x_i, \ldots, x_n)}{h}
\]

if this limit exists. In other words, the partial derivative of \( f \) with respect to \( x_i \) at a point is the slope of the tangent line to the curve you get by fixing every variable except \( x_i \) and graphing \( f \) as a function of \( x_i \) alone.

\subsection{Geometric Interpretation}
In the context of a function of two variables, \( f(x, y) \), the partial derivatives \( \frac{\partial f}{\partial x} \) and \( \frac{\partial f}{\partial y} \) represent the slope of the tangent plane to the surface defined by \( f \) in the direction of the \( x \)-axis and \( y \)-axis respectively.

\subsection{Higher Order Partial Derivatives}
If the partial derivatives of a function are themselves differentiable, we can take their partial derivatives. These are known as higher-order partial derivatives. For instance, the second-order partial derivative of \( f \) with respect to \( x_i \) and then \( x_j \) is denoted as \( \frac{\partial^2 f}{\partial x_j \partial x_i} \).

